\chapter{Simulation der Fahrt und Sensoren}

In diesem Kapitel soll erklärt werden wie wir die Fahrt durch die Straße simuliert haben, wie die Detektion der Objekte im Programm funktioniert und wie die Sensorfunktionen umgesetzt wurden.

\section{Animation der Fahrt}

Die Grundidee der Straßensimulation ist, das Auto zu jedem Zeitpunkt als Mittelpunkt unseres Koordinatensystems anzusehen. Das bedeutet, dass die Umfeld-Koordinaten (Straße, Objekte usw.)  zu jedem Zeitpunkt neu berechnet werden müssen. Hierbei zeigt die Fahrzeugfront immer in y-Richtung.

\subsection{Koordinatentransformation}





