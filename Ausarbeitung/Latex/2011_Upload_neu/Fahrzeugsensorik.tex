\NeedsTeXFormat{LaTeX2e}
\documentclass[a4paper,10pt,bibtotoc,twoside,openright,pointlessnumbers,normalheadings,DIV=15,BCOR25mm
%,draft
]{scrbook}
\KOMAoptions{DIV=last}


\pagestyle{headings}
\usepackage[ngerman]{babel}
%\usepackage[latin1]{inputenc}
%\usepackage[applemac]{inputenc} % Mac-Nutzer
\usepackage[utf8]{inputenc}
\usepackage[T1]{fontenc}
\renewcommand{\sfdefault}{phv}
\renewcommand{\rmdefault}{phv}
\renewcommand{\ttdefault}{pcr}
\usepackage{graphicx}
\usepackage{verbatim}
\usepackage{tabularx}
\usepackage{url}
\usepackage{color}
\usepackage{amssymb}
\usepackage{setspace}
\usepackage{listings}
\lstset{language=Java,
  showstringspaces=false,
  frame=single,
  numbers=left,
  basicstyle=\ttfamily,
  numberstyle=\tiny}

% hier Namen etc. einsetzen
\newcommand{\fullname}{David Kreuzer, \ Daniel Jung \ und \ Jonas Mattes}
\newcommand{\email}{david.kreuzer@uni-ulm.de, \ daniel.jung@uni-ulm.de, \ jonas.mattes@uni-ulm.de}
\newcommand{\titel}{Projekt CSE - Simulation der Fahrzeugsensorik in Matlab}
%\newcommand{\titel}{Titel der Arbeit}
\newcommand{\jahr}{2014}
\newcommand{\matnr}{719866}
\newcommand{\gutachterA}{~Dr.~Otto Löhlein}
\newcommand{\gutachterB}{Prof.~Dr.~Un Leserlich}
\newcommand{\betreuer}{Sebastian Krebs}
% hier richtige Fakultät auswählen
\newcommand{\fakultaet}{Ingenieurwissenschaften\\und Informatik}
%\newcommand{\fakultaet}{Mathematik und\\Wirtschaftswissenschaften}
%\newcommand{\fakultaet}{Naturwissenschaften}
%\newcommand{\fakultaet}{Medizin}
% nun noch unten das Institut einsetzen
\newcommand{\institut}{Institut für Numerische Mathematik}

%color in tables
\usepackage{colortbl}
\definecolor{Gray}{rgb}{0.80784, 0.86667, 0.90196} %dunkelblau
\definecolor{Lightgray}{rgb}{0.9176, 0.95, 0.95686} %hellblau
\definecolor{Akzent}{rgb}{0.6627, 0.63529, 0.55294} %akzentfarbe
\setlength{\arrayrulewidth}{0.1pt}

\clubpenalty10000
\widowpenalty10000

\setlength{\parindent}{0pt}
\setlength{\parskip}{1.4ex plus 0.35ex minus 0.3ex}

% Tiefe, bis zu der Überschriften in das Inhaltsverzeichnis kommen
\setcounter{tocdepth}{3}

\pdfinfo{
  /Author (\fullname)
  /Title (\titel)
  /Producer     (pdfeTex 3.14159-1.30.6-2.2)
  /Keywords ()
}

\usepackage{hyperref}
\hypersetup{
pdftitle=\titel,
pdfauthor=\fullname,
pdfsubject={Projektarbeit},
pdfproducer={pdfeTex 3.14159-1.30.6-2.2},
colorlinks=false,
pdfborder=0 0 0	% keine Box um die Links!
}

%Trennungsregeln
\hyphenation{Sil-ben-trenn-ung}

\begin{document}
\frontmatter

% Titelseite
\thispagestyle{empty}
\begin{addmargin*}[4mm]{-10mm}

\includegraphics[height=1.8cm]{images/unilogo_bild}
\hfill
\includegraphics[height=1.8cm]{images/unilogo_wort}\\[1em]

{\footnotesize
%{\bfseries Universität Ulm} \textbar ~89069 Ulm \textbar ~Germany
\hspace*{116mm}\parbox[t]{38mm}{\bfseries Fakultät für\\
\fakultaet\\
% TODO hier Institut anpassen
\mdseries \institut}\\[2cm]

\parbox{140mm}{\bfseries \huge \titel}\\[0.5em]
{\footnotesize Projektarbeit an der Universität Ulm}\\[3em]

{\footnotesize \bfseries Vorgelegt von:}\\
{\footnotesize \fullname\\\email}\\[2em]
{\footnotesize \bfseries Gutachter:}\\                     
{\footnotesize\gutachterA\\
}\\[2em]
{\footnotesize \bfseries Betreuer:}\\ 
{\footnotesize\betreuer}\\\\
{\footnotesize\jahr}
}
\end{addmargin*}


% Impressum
\clearpage
\thispagestyle{empty}
{ \small
  \flushleft
  Fassung \today \\\vfill
  \copyright~\jahr~\fullname\\[0.5em]
% Falls keine Lizenz gewünscht wird bitte den folgenden Text entfernen.
% Die Lizenz erlaubt es zu nichtkommerziellen Zwecken die Arbeit zu
% vervielfältigen und Kopien zu machen. Dabei muss aber immer der Autor
% angegeben werden. Eine kommerzielle Verwertung ist für den Autor
% weiter möglich.

}


% ab hier Zeilenabstand 1,4 fach 10pt/14pt
\setstretch{1.4}

\tableofcontents

\mainmatter
\chapter{Einleitung}

Diese kleine Einleitung soll dem Nutzer helfen selbst die eigene Arbeit mit \LaTeX zu schreiben. Sie enth"alt Beispiele zu den wichtigsten Themen .


\section{Struktur}

Für diese Arbeit lassen sich als "Uberschriften die "Uberschriften in verschiedenen Stufen verwenden.


\chapter{Erstellung der Strecke und Umgebung}

\section{Aufbereitung der Kartendaten}
\section{Interpolation}
\section{Objektplatzierung}

\chapter{Die Fahrt}



\section{Step-Funktion}
\section{Animation der Fahrt}

Die Grundidee der Straßensimulation ist, das Auto zu jedem Zeitpunkt als Mittelpunkt unseres Koordinatensystems anzusehen. Das bedeutet, dass die Umfeld-Koordinaten (Straße, Objekte usw.)  zu jedem Zeitpunkt neu berechnet werden müssen. Hierbei zeigt die Fahrzeugfront immer in y-Richtung.

\subsection{Koordinatentransformation}





\chapter{Visualisierung in 3D}

\section{Generelle Einführung in Virtual Reality}
%Hier wird eine kurze Einführung in Virtual Reality geschrieben
Die \glqq Virtual Reality Toolbox \grqq \ ist eine Erweiterung, herausgegeben von Mathworks, die es ermöglicht Simulationsergebnisse in 3D darzustellen. \\

Gerade für die Präsentation der eigenen Arbeit, spielt die Visualisierung von Ergebnissen heutzutage eine große Rolle. Hierzu bietet die Toolbox eine gute Möglichkeit.\\ 
Im folgenden Abschnitt möchte ich kurz auf die Erstellung einer solchen "Virtuellen Realität" eingehen und anschließend speziell auf die Erstellung unserer Strecke und deren Umgebung eingehen. \\ \\
Prinzipiell ist der Umgang mit   \glqq Matlab Virtual Reality \grqq \
recht intuitiv. Gearbeitet wird mit VRML (Virtual Reality Modeling Language) Dateien. Bei einfachen Darstellungen muss der Ersteller jedoch keinerlei Kenntnisse über die VRM- Sprache  besitzen. 
Mathworks hat zur Erstellung der \glqq Welt \grqq \ mehrere Editoren vorinstalliert. Hierbei wird dem Benutzer über eine übersichtliche Oberfläche geholfen sich zurechtzufinden und beispielsweise Objekte durch einfaches drag \& drop zu platzieren. Da der \glqq VRealm Builder \grqq \ aus meiner Sicht der übersichtlichste Editor ist, werde ich im Folgenden mit diesem Arbeiten. 
\subsection{VRealm Builder}
Der \glqq VRealm Builder \grqq \ ist in Matlab zwar schon vorhanden, allerdings ist dessen Benutzung nicht voreingestellt. 
\paragraph{Einrichtung} \ \\ \\
Um zu überprüfen ob die VR - Toolbox richtig installiert ist kann folgender Befehl in Matlab getippt werden: \begin{itemize}
	\item[] \textbf{vrinstall -check}
\end{itemize}
 Wenn der VRealm Builder richtig installiert ist, sollten folgende Befehle im Commandwindow erscheinen
 Befehl in Matlab getippt werden: \begin{itemize}
 	\item[] \textbf{VRML viewer : installed}
 	 \item[] \textbf{VRML editor : installed}
 \end{itemize}
 Wenn der Viewer oder Editor noch nicht istalliert sind, kann die Installation mithilfe des Befehls :
  \begin{itemize}
  	\item[] \textbf{vrinstall -install}
    \end{itemize}
 angestoßen werden.\\ \\
 Nachdem die Installation abgeschlossen ist kann der Editor über eine .exe-Datei unter folgendem Pfad gestartet werden : 
  \begin{itemize}
  	\item[] \textbf{[Matlab installation folder]\path{\toolbox\sl3d\vrealm\program}}
  \end{itemize}
  \paragraph{Editoroberfläche} \ \\ \\

\section{Erstellung der Strecke}
\section{Automatisches einfügen und platzieren von Objekten (Aerospace Toolbox in Matlab)}
\section{Animation des Fahrzeugs}



\chapter{Simulation der Sensoren}



\section{Allgemeine Inforamtionen zu den betrachteten Sensoren}










% hier weitere Kapitel einbinden


\appendix
% hier Anhänge einbinden
\input{sources}

\backmatter

\bibliographystyle{plaindin} % Nummern und alphabetisch sortiert
%\bibliographystyle{alphadin} % Buchstaben und sortiert
%\bibliographystyle{abbrvdin} % Nummern und abgekürzte Namen
%\bibliographystyle{unsrtdin} % Nummern und unsortiert
\bibliography{bibliography}


\clearpage
\thispagestyle{empty}

Name: \fullname \hfill Matrikelnummer: \matnr \vspace{2cm}

\minisec{Erklärung}

Ich erkläre, dass ich die Arbeit selbständig verfasst und keine anderen als die angegebenen Quellen und Hilfsmittel verwendet habe.\vspace{2cm}

Ulm, den \dotfill

\hspace{10cm} {\footnotesize \fullname}
\end{document}
