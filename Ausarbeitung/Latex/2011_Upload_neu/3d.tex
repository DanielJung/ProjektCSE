\chapter{Visualisierung in 3D}

\section{Generelle Einführung in Virtual Reality}
%Hier wird eine kurze Einführung in Virtual Reality geschrieben
Die \glqq Virtual Reality Toolbox \grqq \ ist eine Erweiterung, herausgegeben von Mathworks, die es ermöglicht Simulationsergebnisse in 3D darzustellen. \\

Gerade für die Präsentation der eigenen Arbeit, spielt die Visualisierung von Ergebnissen heutzutage eine große Rolle. Hierzu bietet die Toolbox eine gute Möglichkeit.\\ 
Im folgenden Abschnitt möchte ich kurz auf die Erstellung einer solchen "Virtuellen Realität" eingehen und anschließend speziell auf die Erstellung unserer Strecke und deren Umgebung eingehen. \\ \\
Prinzipiell ist der Umgang mit   \glqq Matlab Virtual Reality \grqq \
recht intuitiv. Gearbeitet wird mit VRML (Virtual Reality Modeling Language) Dateien. Bei einfachen Darstellungen muss der Ersteller jedoch keinerlei Kenntnisse über die VRM- Sprache  besitzen. 
Mathworks hat zur Erstellung der \glqq Welt \grqq \ mehrere Editoren vorinstalliert. Hierbei wird dem Benutzer über eine übersichtliche Oberfläche geholfen sich zurechtzufinden und beispielsweise Objekte durch einfaches drag \& drop zu platzieren. Da der \glqq VRealm Builder \grqq \ aus meiner Sicht der übersichtlichste Editor ist, werde ich im Folgenden mit diesem Arbeiten. 
\subsection{VRealm Builder}
Der \glqq VRealm Builder \grqq \ ist in Matlab zwar schon vorhanden, allerdings ist dessen Benutzung nicht voreingestellt. 
\paragraph{Einrichtung} \ \\ \\
Um zu überprüfen ob die VR - Toolbox richtig installiert ist kann folgender Befehl in Matlab getippt werden: \begin{itemize}
	\item[] \textbf{vrinstall -check}
\end{itemize}
 Wenn der VRealm Builder richtig installiert ist, sollten folgende Befehle im Commandwindow erscheinen
 Befehl in Matlab getippt werden: \begin{itemize}
 	\item[] \textbf{VRML viewer : installed}
 	 \item[] \textbf{VRML editor : installed}
 \end{itemize}
 Wenn der Viewer oder Editor noch nicht istalliert sind, kann die Installation mithilfe des Befehls :
  \begin{itemize}
  	\item[] \textbf{vrinstall -install}
    \end{itemize}
 angestoßen werden.\\ \\
 Nachdem die Installation abgeschlossen ist kann der Editor über eine .exe-Datei unter folgendem Pfad gestartet werden : 
  \begin{itemize}
  	\item[] \textbf{[Matlab installation folder]\path{\toolbox\sl3d\vrealm\program}}
  \end{itemize}
  \paragraph{Editoroberfläche} \ \\ \\

\section{Erstellung der Strecke}
\section{Automatisches einfügen und platzieren von Objekten (Aerospace Toolbox in Matlab)}
\section{Animation des Fahrzeugs}


