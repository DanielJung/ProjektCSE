\chapter{Erstellung der Strecke und Umgebung}

\section{Aufbereitung der Kartendaten}
\section{Interpolation}
\section{Objektplatzierung}

\chapter{Die Fahrt}



\section{Step-Funktion}
\section{Animation der Fahrt}

Die Grundidee der Straßensimulation ist, das Auto zu jedem Zeitpunkt als Mittelpunkt unseres Koordinatensystems anzusehen. Das bedeutet, dass die Umfeld-Koordinaten (Straße, Objekte usw.)  zu jedem Zeitpunkt neu berechnet werden müssen. Hierbei zeigt die Fahrzeugfront immer in y-Richtung.

\subsection{Koordinatentransformation}




